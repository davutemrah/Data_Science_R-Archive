% Options for packages loaded elsewhere
\PassOptionsToPackage{unicode}{hyperref}
\PassOptionsToPackage{hyphens}{url}
%
\documentclass[
]{article}
\usepackage{amsmath,amssymb}
\usepackage{lmodern}
\usepackage{ifxetex,ifluatex}
\ifnum 0\ifxetex 1\fi\ifluatex 1\fi=0 % if pdftex
  \usepackage[T1]{fontenc}
  \usepackage[utf8]{inputenc}
  \usepackage{textcomp} % provide euro and other symbols
\else % if luatex or xetex
  \usepackage{unicode-math}
  \defaultfontfeatures{Scale=MatchLowercase}
  \defaultfontfeatures[\rmfamily]{Ligatures=TeX,Scale=1}
\fi
% Use upquote if available, for straight quotes in verbatim environments
\IfFileExists{upquote.sty}{\usepackage{upquote}}{}
\IfFileExists{microtype.sty}{% use microtype if available
  \usepackage[]{microtype}
  \UseMicrotypeSet[protrusion]{basicmath} % disable protrusion for tt fonts
}{}
\makeatletter
\@ifundefined{KOMAClassName}{% if non-KOMA class
  \IfFileExists{parskip.sty}{%
    \usepackage{parskip}
  }{% else
    \setlength{\parindent}{0pt}
    \setlength{\parskip}{6pt plus 2pt minus 1pt}}
}{% if KOMA class
  \KOMAoptions{parskip=half}}
\makeatother
\usepackage{xcolor}
\IfFileExists{xurl.sty}{\usepackage{xurl}}{} % add URL line breaks if available
\IfFileExists{bookmark.sty}{\usepackage{bookmark}}{\usepackage{hyperref}}
\hypersetup{
  pdftitle={Final Project Part I},
  hidelinks,
  pdfcreator={LaTeX via pandoc}}
\urlstyle{same} % disable monospaced font for URLs
\usepackage[margin=1in]{geometry}
\usepackage{color}
\usepackage{fancyvrb}
\newcommand{\VerbBar}{|}
\newcommand{\VERB}{\Verb[commandchars=\\\{\}]}
\DefineVerbatimEnvironment{Highlighting}{Verbatim}{commandchars=\\\{\}}
% Add ',fontsize=\small' for more characters per line
\usepackage{framed}
\definecolor{shadecolor}{RGB}{248,248,248}
\newenvironment{Shaded}{\begin{snugshade}}{\end{snugshade}}
\newcommand{\AlertTok}[1]{\textcolor[rgb]{0.94,0.16,0.16}{#1}}
\newcommand{\AnnotationTok}[1]{\textcolor[rgb]{0.56,0.35,0.01}{\textbf{\textit{#1}}}}
\newcommand{\AttributeTok}[1]{\textcolor[rgb]{0.77,0.63,0.00}{#1}}
\newcommand{\BaseNTok}[1]{\textcolor[rgb]{0.00,0.00,0.81}{#1}}
\newcommand{\BuiltInTok}[1]{#1}
\newcommand{\CharTok}[1]{\textcolor[rgb]{0.31,0.60,0.02}{#1}}
\newcommand{\CommentTok}[1]{\textcolor[rgb]{0.56,0.35,0.01}{\textit{#1}}}
\newcommand{\CommentVarTok}[1]{\textcolor[rgb]{0.56,0.35,0.01}{\textbf{\textit{#1}}}}
\newcommand{\ConstantTok}[1]{\textcolor[rgb]{0.00,0.00,0.00}{#1}}
\newcommand{\ControlFlowTok}[1]{\textcolor[rgb]{0.13,0.29,0.53}{\textbf{#1}}}
\newcommand{\DataTypeTok}[1]{\textcolor[rgb]{0.13,0.29,0.53}{#1}}
\newcommand{\DecValTok}[1]{\textcolor[rgb]{0.00,0.00,0.81}{#1}}
\newcommand{\DocumentationTok}[1]{\textcolor[rgb]{0.56,0.35,0.01}{\textbf{\textit{#1}}}}
\newcommand{\ErrorTok}[1]{\textcolor[rgb]{0.64,0.00,0.00}{\textbf{#1}}}
\newcommand{\ExtensionTok}[1]{#1}
\newcommand{\FloatTok}[1]{\textcolor[rgb]{0.00,0.00,0.81}{#1}}
\newcommand{\FunctionTok}[1]{\textcolor[rgb]{0.00,0.00,0.00}{#1}}
\newcommand{\ImportTok}[1]{#1}
\newcommand{\InformationTok}[1]{\textcolor[rgb]{0.56,0.35,0.01}{\textbf{\textit{#1}}}}
\newcommand{\KeywordTok}[1]{\textcolor[rgb]{0.13,0.29,0.53}{\textbf{#1}}}
\newcommand{\NormalTok}[1]{#1}
\newcommand{\OperatorTok}[1]{\textcolor[rgb]{0.81,0.36,0.00}{\textbf{#1}}}
\newcommand{\OtherTok}[1]{\textcolor[rgb]{0.56,0.35,0.01}{#1}}
\newcommand{\PreprocessorTok}[1]{\textcolor[rgb]{0.56,0.35,0.01}{\textit{#1}}}
\newcommand{\RegionMarkerTok}[1]{#1}
\newcommand{\SpecialCharTok}[1]{\textcolor[rgb]{0.00,0.00,0.00}{#1}}
\newcommand{\SpecialStringTok}[1]{\textcolor[rgb]{0.31,0.60,0.02}{#1}}
\newcommand{\StringTok}[1]{\textcolor[rgb]{0.31,0.60,0.02}{#1}}
\newcommand{\VariableTok}[1]{\textcolor[rgb]{0.00,0.00,0.00}{#1}}
\newcommand{\VerbatimStringTok}[1]{\textcolor[rgb]{0.31,0.60,0.02}{#1}}
\newcommand{\WarningTok}[1]{\textcolor[rgb]{0.56,0.35,0.01}{\textbf{\textit{#1}}}}
\usepackage{graphicx}
\makeatletter
\def\maxwidth{\ifdim\Gin@nat@width>\linewidth\linewidth\else\Gin@nat@width\fi}
\def\maxheight{\ifdim\Gin@nat@height>\textheight\textheight\else\Gin@nat@height\fi}
\makeatother
% Scale images if necessary, so that they will not overflow the page
% margins by default, and it is still possible to overwrite the defaults
% using explicit options in \includegraphics[width, height, ...]{}
\setkeys{Gin}{width=\maxwidth,height=\maxheight,keepaspectratio}
% Set default figure placement to htbp
\makeatletter
\def\fps@figure{htbp}
\makeatother
\setlength{\emergencystretch}{3em} % prevent overfull lines
\providecommand{\tightlist}{%
  \setlength{\itemsep}{0pt}\setlength{\parskip}{0pt}}
\setcounter{secnumdepth}{-\maxdimen} % remove section numbering
\ifluatex
  \usepackage{selnolig}  % disable illegal ligatures
\fi

\title{Final Project Part I}
\author{}
\date{\vspace{-2.5em}}

\begin{document}
\maketitle

{
\setcounter{tocdepth}{2}
\tableofcontents
}
\hypertarget{part-1---simulation}{%
\subsection{Part 1 - Simulation}\label{part-1---simulation}}

\hypertarget{author-davut-emrah-ayan}{%
\subparagraph{Author: Davut Emrah Ayan}\label{author-davut-emrah-ayan}}

Abstract: In this project, first, I will investigate the exponential
distribution and compare it with Central Limit theorem.

\hypertarget{data-generating-process}{%
\subparagraph{Data generating process}\label{data-generating-process}}

I generated 1000 samples of size 40 from exponential distribution of
rate = 0.2

\begin{Shaded}
\begin{Highlighting}[]
\FunctionTok{rm}\NormalTok{(}\AttributeTok{list =} \FunctionTok{ls}\NormalTok{())}
\FunctionTok{set.seed}\NormalTok{(}\DecValTok{12345}\NormalTok{)}
\NormalTok{simN }\OtherTok{=} \DecValTok{1000}
\NormalTok{n }\OtherTok{=} \DecValTok{40}
\NormalTok{lambda }\OtherTok{=} \FloatTok{0.2}

\NormalTok{X }\OtherTok{=} \FunctionTok{rexp}\NormalTok{(}\AttributeTok{n =}\NormalTok{ n}\SpecialCharTok{*}\NormalTok{simN, }\AttributeTok{rate =}\NormalTok{ lambda)}
\NormalTok{sampleX }\OtherTok{=} \FunctionTok{matrix}\NormalTok{(X, }\AttributeTok{nrow =}\NormalTok{ simN)}
\end{Highlighting}
\end{Shaded}

\hypertarget{show-the-sample-mean-and-compare-it-to-the-theoretical-mean-of-the-distribution.}{%
\subsubsection{1. Show the sample mean and compare it to the theoretical
mean of the
distribution.}\label{show-the-sample-mean-and-compare-it-to-the-theoretical-mean-of-the-distribution.}}

Distribution of simulated means

I put 1000 simulated means in meanX vector. I find that mean of sample
means is 4.971972. On the other hand, theoretical mean is
\(1/\lambda = 1/0.2 = 5\).

\begin{Shaded}
\begin{Highlighting}[]
\NormalTok{meanX }\OtherTok{=} \FunctionTok{as.data.frame}\NormalTok{(}\FunctionTok{apply}\NormalTok{(sampleX, }\DecValTok{1}\NormalTok{, mean))}
\FunctionTok{names}\NormalTok{(meanX) }\OtherTok{=} \StringTok{"means"}
\FunctionTok{mean}\NormalTok{(meanX}\SpecialCharTok{$}\NormalTok{means)}
\end{Highlighting}
\end{Shaded}

\begin{verbatim}
## [1] 4.971972
\end{verbatim}

Below I show the distribution of simulated sample means in the histogram
and compare means with the theoretical Mean represented by dashed line.

\begin{Shaded}
\begin{Highlighting}[]
\FunctionTok{library}\NormalTok{(ggplot2)}
\FunctionTok{ggplot}\NormalTok{(meanX, }\FunctionTok{aes}\NormalTok{(means)) }\SpecialCharTok{+} 
    \FunctionTok{geom\_histogram}\NormalTok{(}\AttributeTok{binwidth =} \FloatTok{0.2}\NormalTok{) }\SpecialCharTok{+} 
    \FunctionTok{geom\_vline}\NormalTok{(}\AttributeTok{xintercept =} \DecValTok{1}\SpecialCharTok{/}\NormalTok{lambda, }\AttributeTok{linetype =} \StringTok{"dashed"}\NormalTok{) }\SpecialCharTok{+}
    \FunctionTok{labs}\NormalTok{(}\AttributeTok{title =} \StringTok{\textquotesingle{}Histogram of Simulated Means\textquotesingle{}}\NormalTok{,}
         \AttributeTok{subtitle =} \StringTok{"Theoretical mean (1/lambda) = 5"}\NormalTok{) }\SpecialCharTok{+} 
    \FunctionTok{theme}\NormalTok{(}\AttributeTok{plot.title =} \FunctionTok{element\_text}\NormalTok{(}\AttributeTok{hjust =} \FloatTok{0.5}\NormalTok{)) }\SpecialCharTok{+}
    \FunctionTok{xlab}\NormalTok{(}\FunctionTok{element\_blank}\NormalTok{())}
\end{Highlighting}
\end{Shaded}

\includegraphics{Final_Project_Ayan_files/figure-latex/hist-1.pdf}

\hypertarget{show-how-variable-the-sample-is-via-variance-and-compare-it-to-the-theoretical-variance-of-the-distribution.}{%
\subsubsection{2. Show how variable the sample is (via variance) and
compare it to the theoretical variance of the
distribution.}\label{show-how-variable-the-sample-is-via-variance-and-compare-it-to-the-theoretical-variance-of-the-distribution.}}

Below I calculated variance of 1000 samples and put them in varianceX
vector.

\begin{Shaded}
\begin{Highlighting}[]
\NormalTok{varianceX }\OtherTok{=} \FunctionTok{as.data.frame}\NormalTok{(}\FunctionTok{apply}\NormalTok{(sampleX, }\DecValTok{1}\NormalTok{, var))}
\FunctionTok{names}\NormalTok{(varianceX) }\OtherTok{=} \StringTok{"variances"}
\end{Highlighting}
\end{Shaded}

Variance (Standard error) of the simulated sample is
\(s_\bar{x}^2 = 0.6157926\)

\begin{Shaded}
\begin{Highlighting}[]
\FunctionTok{var}\NormalTok{(meanX}\SpecialCharTok{$}\NormalTok{means)}
\end{Highlighting}
\end{Shaded}

\begin{verbatim}
## [1] 0.6157926
\end{verbatim}

and theoretical variance of the sample is
\(var(\bar{x}) = \sigma_x^2/n = 0.625\)

\begin{Shaded}
\begin{Highlighting}[]
\NormalTok{(}\DecValTok{1}\SpecialCharTok{/}\NormalTok{lambda)}\SpecialCharTok{\^{}}\DecValTok{2}\SpecialCharTok{/}\DecValTok{40}
\end{Highlighting}
\end{Shaded}

\begin{verbatim}
## [1] 0.625
\end{verbatim}

Below I show distribution of simulated sample variances and compare with
the theoretical Variance. Theoretical variance is
\((1/\lambda)^2 = (1/0.2)^2 = 25\). In the histogram, dashed line
represents the theoretical variance of 5.

Distribution of simulated variances

\begin{Shaded}
\begin{Highlighting}[]
\FunctionTok{library}\NormalTok{(ggplot2)}
\NormalTok{g2 }\OtherTok{\textless{}{-}} \FunctionTok{ggplot}\NormalTok{(varianceX, }\FunctionTok{aes}\NormalTok{(variances)) }\SpecialCharTok{+} 
    \FunctionTok{geom\_histogram}\NormalTok{(}\AttributeTok{binwidth =} \FloatTok{2.5}\NormalTok{) }\SpecialCharTok{+} 
    \FunctionTok{geom\_vline}\NormalTok{(}\AttributeTok{xintercept =}\NormalTok{ ((}\DecValTok{1}\SpecialCharTok{/}\NormalTok{lambda)}\SpecialCharTok{/}\FunctionTok{sqrt}\NormalTok{(}\DecValTok{40}\NormalTok{))}\SpecialCharTok{\^{}}\DecValTok{2}\NormalTok{, }\AttributeTok{linetype =} \StringTok{"dashed"}\NormalTok{) }\SpecialCharTok{+}
    \FunctionTok{labs}\NormalTok{(}\AttributeTok{title =} \StringTok{\textquotesingle{}Histogram of Simulated Variances\textquotesingle{}}\NormalTok{,}
         \AttributeTok{subtitle =} \FunctionTok{paste}\NormalTok{(}\StringTok{"Theoretical variance is (1/lambda)\^{}2) = 25"}\NormalTok{)) }\SpecialCharTok{+} 
    \FunctionTok{theme}\NormalTok{(}\AttributeTok{plot.title =} \FunctionTok{element\_text}\NormalTok{(}\AttributeTok{hjust =} \FloatTok{0.5}\NormalTok{)) }\SpecialCharTok{+}
    \FunctionTok{xlab}\NormalTok{(}\FunctionTok{element\_blank}\NormalTok{())}
\NormalTok{g2}
\end{Highlighting}
\end{Shaded}

\includegraphics{Final_Project_Ayan_files/figure-latex/histvar-1.pdf}

\hypertarget{show-that-the-distribution-is-approximately-normal.}{%
\subsubsection{3. Show that the distribution is approximately
normal.}\label{show-that-the-distribution-is-approximately-normal.}}

Central limit theorem suggests that distribution of samples from an
unknown distribution are approximated to normal distribution as sample
sizes increase. In this project, I draw 1000 samples of size 40 from
exponential distribution where rate = 0.2.

By CLT, I expect that distribution of simulated data approximates the
Normal Distribution where \(\mu = E(\bar{x})\), and \(Var = s^2/n\) in
this project.

Both distributions closely overlap.

\begin{Shaded}
\begin{Highlighting}[]
\FunctionTok{ggplot}\NormalTok{(}\AttributeTok{data =}\NormalTok{ meanX, }\FunctionTok{aes}\NormalTok{(}\AttributeTok{x =}\NormalTok{ means)) }\SpecialCharTok{+} 
    \FunctionTok{geom\_density}\NormalTok{() }\SpecialCharTok{+}
    \FunctionTok{stat\_function}\NormalTok{(}\AttributeTok{fun =}\NormalTok{ dnorm,}
                  \AttributeTok{args =} \FunctionTok{list}\NormalTok{(}\AttributeTok{mean =}\NormalTok{ (}\DecValTok{1}\SpecialCharTok{/}\NormalTok{lambda), }\AttributeTok{sd =}\NormalTok{ (}\DecValTok{1}\SpecialCharTok{/}\NormalTok{lambda)}\SpecialCharTok{/}\FunctionTok{sqrt}\NormalTok{(}\DecValTok{40}\NormalTok{)),}
                  \AttributeTok{colour =} \StringTok{"red"}\NormalTok{, }\AttributeTok{size =} \DecValTok{1}\NormalTok{) }\SpecialCharTok{+}
    \FunctionTok{geom\_density}\NormalTok{(}\AttributeTok{colour=}\StringTok{"blue"}\NormalTok{, }\AttributeTok{size=}\DecValTok{1}\NormalTok{, }\AttributeTok{alpha =}\NormalTok{ .}\DecValTok{2}\NormalTok{) }\SpecialCharTok{+}
    \FunctionTok{geom\_vline}\NormalTok{(}\AttributeTok{xintercept =} \FunctionTok{c}\NormalTok{(}\DecValTok{1}\SpecialCharTok{/}\NormalTok{lambda, }\FunctionTok{mean}\NormalTok{(meanX}\SpecialCharTok{$}\NormalTok{means)), }\AttributeTok{linetype =} \StringTok{"dashed"}\NormalTok{, }\AttributeTok{color =} \FunctionTok{c}\NormalTok{(}\StringTok{"red"}\NormalTok{, }\StringTok{"blue"}\NormalTok{)) }\SpecialCharTok{+}
    \FunctionTok{labs}\NormalTok{(}\AttributeTok{title =} \StringTok{\textquotesingle{}Density Plots of Simulated Means and Normal distribution\textquotesingle{}}\NormalTok{,}
         \AttributeTok{subtitle =} \StringTok{"Theoretical mean (1/lambda) = 5"}\NormalTok{) }\SpecialCharTok{+} 
    \FunctionTok{theme}\NormalTok{(}\AttributeTok{plot.title =} \FunctionTok{element\_text}\NormalTok{(}\AttributeTok{hjust =} \FloatTok{0.5}\NormalTok{)) }\SpecialCharTok{+}
    \FunctionTok{xlab}\NormalTok{(}\FunctionTok{element\_blank}\NormalTok{()) }
\end{Highlighting}
\end{Shaded}

\includegraphics{Final_Project_Ayan_files/figure-latex/density-1.pdf}

\end{document}
